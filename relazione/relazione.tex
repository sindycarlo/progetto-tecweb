\documentclass[12pt]{article}
\usepackage{hyperref}
\usepackage{graphicx}
\usepackage[font=small,labelfont=bf]{caption}
\title{Progetto di fine corso}
\date{17/05/2016}
\author{Alessio Luca,Carlo Sindico}

\begin{document}
	\pagenumbering{arabic}
	
	\begin{titlepage}
		\newcommand{\HRule}{\rule{\linewidth}{0.5mm}}%linea orizzontale
		\center
		
		\textsc{\LARGE Universit\`a degli Studi di Padova}\\[1.5cm] 
		\textsc{\Large Laurea in Informatica}\\[0.5cm]
		\textsc{\large Corso di Tecnologie Web}\\[0.5cm]
		\textsc{\large Progetto di fine corso}\\[0.5cm]
		
		%----------------------------------------------------------------------------------------
		%	TITLE SECTION
		%----------------------------------------------------------------------------------------
		
		\HRule \\[0.4cm]
		{ \huge  2FORCHETTE}\\[0.3cm] 
		\HRule \\[0.4cm]
		
		
		%----------------------------------------------------------------------------------------
		%	AUTHOR SECTION
		%----------------------------------------------------------------------------------------
		
		\begin{minipage}{0.3\textwidth}
			\begin{flushleft} \large
				\emph{Studente:}\\
				Luca \textsc{Alessio} % Your name
			\end{flushleft}
		\end{minipage}
		~
		\begin{minipage}{0.3\textwidth}
			\begin{flushright} \large
				\emph{Matricola:} \\
				\textsc{1070690} % Supervisor's Name
			\end{flushright}
		\end{minipage}\\[2cm]
		
			\begin{minipage}{0.3\textwidth}
				\begin{flushleft} \large
					\emph{Studente:}\\
					Carlo \textsc{Sindico} % Your name
				\end{flushleft}
			\end{minipage}
			~
			\begin{minipage}{0.3\textwidth}
				\begin{flushright} \large
					\emph{Matricola:} \\
					\textsc{1069322} % Supervisor's Name
				\end{flushright}
			\end{minipage}\\[2cm]
			
		%----------------------------------------------------------------------------------------
		%	INFORMATION WEBSITE
		%----------------------------------------------------------------------------------------
		
		\textsc{\Large Informazioni sul sito:}\\[0.3cm]	
		\textsc{http://tecnologie-web.studenti.math.unipd.it/tecweb/~csindico/}\\[1cm]
		
		%----------------------------------------------------------------------------------------
		%	DATI LOGIN
		%----------------------------------------------------------------------------------------
		
			\textsc{\Large Dati login:}\\[0.3cm]
			\textsc{ Username:Admin}\\[0.1mm]
			\textsc{ Password:Admin}\\[0.1mm]
		\vfill
	\end{titlepage}
	
	\newpage
	\renewcommand{\contentsname}{Indice}
	\tableofcontents
	
	
	\newpage
	\pagenumbering{arabic}
	
	\section{Abstract}
	\begin{itemize}
		\item Il progetto si propone di offrire  alla maggior parte degli utenti  un sito che ha come contenuto un insieme di ricette. 
			
		Il sito ha un scopo principalmente informativo, infatti riporta tutte le informazioni reltaive a ciascuna ricetta. Le ricette sono classificate in Primi piatti, Secondi piatti, Antipasti e Dessert. L'utente ha la possibilità di proporre una nuova ricetta compilando un apposito form, e un lato amministratore ha il pieno controllo nell'accettare o rifiutare le ricette proposte. Si è trovato utile inserire una sezione commenti, che permette all'utente di chiedere ulteriori informazioni e inviare feedback scrivendo direttamente nel sito, e l'amministratore ha la possibilità di rimuovere i commenti inopportuni.
		
		Il sito è stato creato con lo scopo di essere visualizzato in internet, quindi si è data importanza alla presentazione e alla sua usabilità, rispettando lo standard W3C. Infatti si è data particolare attenzione nella separazione tra struttura presentazione e comportamento con le regole di accessibilità richieste.
	\end{itemize}
		\section{Utenti destinatari}
		\begin{itemize}
			\item
		\end{itemize}
	
	
\end{document}